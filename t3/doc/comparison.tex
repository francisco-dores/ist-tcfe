\section{Comparison between Octave and Ngspice results}
\label{sec:comparison}

\par Table \ref{tab:envelope} shows, on the left, the ripple voltage and the
average output voltage of the evelope detector, which corresponds to the voltage
on node 4, found by Octave, and, on the right, the same variables, but with the
simulation results, by Ngspice, all for the circuit under analysis.

\begin{table}[htb!]
  \centering
  \begin{tabular}{|l|r|}
      \hline    
      {\bf Name} & {\bf Value [V]} \\ \hline
      $Ripple_{envelope}$ & $0.002240$ \\ \hline 
$Average_{envelope}$ & $22.498882$ \\ \hline 

  \end{tabular}
\quad
  \begin{tabular}{|l|r|}
    \hline    
    {\bf Name} & {\bf Value [V]} \\ \hline
    maximum(v(4))-minimum(v(4)) & 7.852210e-02\\ \hline
mean(v(4)) & 2.276490e+02\\ \hline

  \end{tabular}
  \caption{Results for the ripple and average output voltages of the envelope detector.}
  \label{tab:envelope}
\end{table}

\par By analysing the result from both the theoretical and simulated models is possible
to observe differences in values. These discrepancies are, in its majority, due to the
non-linearity of the diodes. Due to this complexity, we are forced to use a simplistic
model in our calculations (the ideal model of the diode), through Octave, while Ngspice uses a more complete model for
the analysis of this circuit. With this, discrepencies in values are expected.

\par Beyond the envelope detector, it can also be compared the ripple voltage and the
average output voltage of the voltage regulator, corresponding to the voltage on node
5. The results from Octave, on the left, and NGspice, on the right, can be observed in
Table \ref{tab:regulator}.

\begin{table}[htb!]
  \centering
  \begin{tabular}{|l|r|}
      \hline    
      {\bf Name} & {\bf Value [V]} \\ \hline
      $Ripple_{regulator}$ & $0.000076$ \\ \hline 
$Average_{regulator}$ & $11.822949$ \\ \hline 

  \end{tabular}
\quad
  \begin{tabular}{|l|r|}
    \hline    
    {\bf Name} & {\bf Value [V]} \\ \hline
    maximum(v(5))-minimum(v(5)) & 1.597796e-04\\ \hline
mean(v(5)) & 1.199992e+01\\ \hline

  \end{tabular}
  \caption{Results for the ripple and average output voltages of the voltage regulator.}
  \label{tab:regulator}
\end{table}

\par Similarly to the envelope detector, discrepancies can also be found on the ripple and
average output voltages of the voltage regulator, by reasons metioned above. The ripple value for the theoretical analysis is much smaller than the one obtained in ngspice, however, the DC value is closer to 12V in the ngspice calculations. The differences in the ripple are expected since we are using a simple incremental analysis to calculate the AC component.

