\section{Comparison between Octave and Ngspice results}
\label{sec:comparison}

\subsection{Analysis for t$<$0}

\par Table \ref{tab:tb0_all} shows, on the left and center, the nodal voltages and the
branch currents, found by Octave, and, on the right, the simulated operating point results,
by Ngspice, all for the circuit under analysis, when t$<$0.

\begin{table}[htb!]
  \centering
  \begin{tabular}{|l|r|}
      \hline    
      {\bf Name} & {\bf Value [V]} \\ \hline
      $V_1$ & 5.12562725920 \\ \hline 
$V_2$ & 4.90389094213 \\ \hline 
$V_3$ & 4.44621543662 \\ \hline 
$V_4$ & 0.00000000000 \\ \hline 
$V_5$ & 4.93496307069 \\ \hline 
$V_6$ & 5.63581571238 \\ \hline 
$V_7$ & -1.97571911988 \\ \hline 
$V_8$ & -2.98274501015 \\ \hline 

  \end{tabular}
\quad
  \begin{tabular}{|l|r|}
    \hline    
    {\bf Name} & {\bf Ampere [A]} \\ \hline
    $I_1$ & 0.00021612262 \\ \hline 
$I_2$ & -0.00022637255 \\ \hline 
$I_3$ & -0.00001024993 \\ \hline 
$I_4$ & 0.00119458862 \\ \hline 
$I_5$ & -0.00022637255 \\ \hline 
$I_6$ & 0.00097846600 \\ \hline 
$I_7$ & 0.00097846600 \\ \hline 
$I_b$ & -0.00022637255 \\ \hline 
$I_c$ & -0.00000000000 \\ \hline 
$I_{V_d}$ & 0.00097846600 \\ \hline 
$I_{V_s}$ & 0.00021612262 \\ \hline 

  \end{tabular}
\quad
  \begin{tabular}{|l|r|}
    \hline    
    {\bf Name} & {\bf Value [A or V]} \\ \hline
    c[i] & 0.000000e+00\\ \hline
gib[i] & -2.26373e-04\\ \hline
r1[i] & 2.161226e-04\\ \hline
r2[i] & -2.26373e-04\\ \hline
r3[i] & -1.02499e-05\\ \hline
r4[i] & 1.194589e-03\\ \hline
r5[i] & -2.26373e-04\\ \hline
r6[i] & 9.784660e-04\\ \hline
r7[i] & 9.784660e-04\\ \hline
v(1) & 5.125627e+00\\ \hline
v(2) & 4.903891e+00\\ \hline
v(3) & 4.446215e+00\\ \hline
v(4) & -1.97572e+00\\ \hline
v(5) & 4.934963e+00\\ \hline
v(6) & 5.635816e+00\\ \hline
v(7) & -1.97572e+00\\ \hline
v(8) & -2.98275e+00\\ \hline

  \end{tabular}
  \caption{Results for the Circuit at $t<0$. A variable followed by [i] or [current] is of type {\em current} and expressed in Ampere; other variables are of type {\it voltage} and expressed in Volt.}
  \label{tab:tb0_all}
\end{table}

\newpage
\subsection{Analysis for t=0}

\par Table \ref{tab:t0_all} shows, on the left, the nodal voltages and the branch
currents, found by Octave, and, on the right, the simulated operating point results,
by Ngspice, both for the circuit under analysis, when $V_s$=0, respectively.

\begin{table}[htb!]
  \centering
  \begin{tabular}{|l|r|}
      \hline    
      {\bf Name} & {\bf Value [V]} \\ \hline
      $V_1$ & 0.00000000000 \\ \hline 
$V_2$ & 0.00000000000 \\ \hline 
$V_3$ & 0.00000000000 \\ \hline 
$V_4$ & 0.00000000000 \\ \hline 
$V_5$ & 0.00000000000 \\ \hline 
$V_6$ & 8.61856072253 \\ \hline 
$V_7$ & 0.00000000000 \\ \hline 
$V_8$ & 0.00000000000 \\ \hline 

  \end{tabular}
\quad
  \begin{tabular}{|l|r|}
    \hline    
    {\bf Name} & {\bf Value [A or V]} \\ \hline
    gib[i] & 6.327120e-18\\ \hline
r1[i] & -6.04063e-18\\ \hline
r2[i] & 6.327120e-18\\ \hline
r3[i] & 2.864858e-19\\ \hline
r4[i] & 1.289989e-18\\ \hline
r5[i] & -2.78376e-03\\ \hline
r6[i] & 1.301043e-18\\ \hline
r7[i] & 2.625374e-18\\ \hline
v(1) & 0.000000e+00\\ \hline
v(2) & 6.197538e-15\\ \hline
v(3) & 1.898958e-14\\ \hline
v(4) & -2.62707e-15\\ \hline
v(5) & 5.329071e-15\\ \hline
v(6) & 8.618561e+00\\ \hline
v(7) & -2.62707e-15\\ \hline
v(8) & -5.32907e-15\\ \hline

  \end{tabular}
  \caption{Results for the Circuit at $t=0$. A variable followed by [i] or [current] is of type {\em current} and expressed in Ampere; other variables are of type {\it voltage} and expressed in Volt.}
  \label{tab:t0_all}
\end{table}




