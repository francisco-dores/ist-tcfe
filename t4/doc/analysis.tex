\section{Theoretical Analysis}
\label{sec:analysis}

\par In this section, we will analyse our theoretical model of the audio amplifier circuit in terms of the voltage gain, input and output impedances, and band width. The audio amplifier, as explained in the introduction is composed by two stages: the gain stage and the output stage. The first one provides a high gain and input impedance, however, it cames with a high output impedance that could degrade the signal transmitted to the load (by voltage division), which is not desirable. Thats why we need the second stage, which has a small gain ($\approx 1$), but provides also a small output impedance. Combining the two circuits, one can obtain a stable audio amplifier with good gain.
 
\subsection{Gain stage}
\label{subsec:gain_stage}

%Incluir imagem do primeiro circuito sem o thévenin aplicado com os dois condensadores

\par We will start by explaining why we need the components shown in the previous image of the gain stage:

\begin{itemize}
\item The resistor $R_{C1}$, connected to the collector terminal of the transistor, is important to obtain a high gain as we will se in the operating point analysis.
\item The resistor $R_{E1}$, connected to the emmiter terminal of the transistor, ensures a temperature stabilization effect by imposing a negative feedback loop. That is important because some paramaters of the transistor are temperature dependent.
\item As we will see next, having the resistor $R_{E1}$ is not good for the AC gain. So, using the capacitor $C_b$ in parallel with the resistor we can make a short circuit for medium/high frequencies of $v_{IN}$, and the effect of the resistor can be neglected for the incremental analysis (because it is in parallel with a short circuit). For the operating point (DC), there is no AC component and the capacitor behaves like a open circuit, therefore, the resistor have the function of stabilization.
\item Because $v_{IN}$ has no DC component, we need a bias circuit, composed by the voltage supplier $V_{CC}$ and two resitors $R_{B1}$ and $R_{B2}$, that ensures the transistor is in the forward active region (FAR), where $V_{CE}>V_{BEON}$. We can simplify this circuit to a Thévenin's equivalent as showed in the next image. 
\item Also, we need another capacitor $C_i$ to block the $0V$ DC component of $v_{IN}$, creating an open circuit for low frequencies (DC analysis). Otherwise, the transistor wouldn't be in FAR. 
\end{itemize}

\subsubsection{Operating point for gain stage}
\label{subsec:OP1}

\par With the simplifications of the capacitors made before for low frequencies and without the voltage $v_{IN}$ (which only has AC component), we can write the following circuit to make the operating point analysis:

%Introduzir imagem do OP1 com thévenin

\par First we reduce the bias circuit to a Thévenin's equivalent obtaining the resistor $R_B$ and the equivalent voltage $V_{eq}$:

\begin{equation}
R_B=R_{B1}||R_{B2}=\frac{R_{B1}R_{B2}}{R_{B1}+R_{B2}
\end{equation}
\begin{equation}
V_{eq}=\frac{R_{B2}}{R_{B1}+R_{B2}V_{CC}
\end{equation}

\par Then we can write KVL to the left mesh, where $V_{BEON}\approx0.7$:
\begin{equation}
-V_{eq}+R_BI_B+V_{BEON}+R_{E1}I_{E1}=0
\end{equation}
\par The currents in the NPN transistor are related by the parameter $\beta _{FN}=178.7$ accordingly to the following equations:
\begin{equation}
I_{E1}=(1+\beta _{FN})I_{B1}
\end{equation}
\begin{equation}
I_{C1}=\beta _{FN}I_{B1}
\end{equation}

\par With the previous equations, we can write a equation to obtain $I_B$ and then, all the other currents in the transistor:
\begin{equation}
I_{B1}=\frac{V_{eq}-V_{ON}}{R_{B1}+(1+\beta _F)R_{E1}}
\end{equation}

\par After that, it's easy to compute the other DC voltages:

\begin{equation}
V_{O1}=V_{CC}-R_{C1}I_{C1}
\end{equation}
\begin{equation}
V_{E1}=R_{E1}I_{E1}
\end{equation}
\begin{equation}
V_{CE}=V_{O1}-V_{E1}
\end{equation}
\par To ensure that the transistor is in the FAR, the next condition must be verified: $V_{CE}>V_{BEON}$. As we can see in the next table with the operating point values this was ensured.

%Por tabela com os valores de OP1
%Comparar valores com simulation 


\subsubsection{Incremental analysis for gain stage}
\label{subsec:inc1}

\par Now, with the values of the operating point we can make the incremental analysis, for medium/high frequencies. As explained before, with this frequencies the capacitor $C_b$ is an open circuit, therefore, we can use the following incremental model (without the resistor $R_{E1}$):

%Imagem incr1

\par The incremental parameters for this NPN transistor are given by:

\begin{equation}
gm1=\frac{I_{C1}}{V_T}
\end{equation}
\begin{equation}
r_{\pi 1}=\frac{\beta _{FN}}{g_{m1}}
\end{equation}
\begin{equation}
r_{o1}=\frac{V_{AFN}}{I_{C1}}
\end{equation}

\par With $V_{AFN}=69.7V$ and $V_T=25e^{-3}V$. Now we can compute the gain of the gain stage given by:

\begin{equation}
A_{V1}=\frac{R_{SB}}{R_S}R_{C1}\frac{-g_{m1}r_{\pi 1}r_{o1}}{(r_{o1}+R_{C1})(R_{SB}+r_{\pi 1})}
\end{equation}
\par With $R_{SB}=\frac {R_BR_S}{R_B+R_S}$

\par To compute the impedances of the gain stage we must consider again the resistor $R_{E1}$ between ground and emitter (not represented in the figure). The input and output impedances are given, respectively, by:

\begin{equation}
Z_{I1}=\frac{1}{\frac{1}{R_B}+\frac{1}{\frac{(r_{o1}+R_{C1}+R_{E1})*(r_{\pi 1}+R_{E1})+g_{m1}*R_{E1}*r_{o1}*r_{\pi 1} - R_{E1}^2)}{r_{o1}+R_{C1}+R_{E1}}}}
\end{equation} 
\begin{equation}
Z_{O1} = \frac{1}{\frac{1}{Z_X}+\frac{1}{R_{C1}}}
\end{equation}
\par Where $Z_X$ is given by:
\begin{equation}
Z_X = r_{o1}*\frac{R_{SB}+\frac{r_{\pi 1}*R_{E1}}{RSB+rpi1+RE1}}{\frac{1}{\frac{1}{r_{o1}}+\frac{1}{rpi1+RSB}+\frac{1}{RE1}+\frac{gm1*rpi1}{rpi1+RSB}}}
\end{equation}

\par In the next table we print the theoretical results of the impedances and gain for the gain stage:

 %tabela impedancias e gain do gain stage
 
 \par As we can see, the output impedance is still very 
 

