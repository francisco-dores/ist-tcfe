\section{Simulation Analysis}
\label{sec:simulation}

\subsection{Operating Point Analysis}

Table~\ref{tab:op} shows the simulated operating point results for the circuit
under analysis. Compared to the theoretical analysis results, one notices the
following differences: describe and explain the differences.

\begin{table}[h]
  \centering
  \begin{tabular}{|l|r|}
    \hline    
    {\bf Name} & {\bf Value [A or V]} \\ \hline
    I_b & -0.000226\\ \hline
I_d & 0.001012\\ \hline
I_{R1} & 0.000216\\ \hline
I_{R2} & -0.000226\\ \hline
I_{R3} & -0.000010\\ \hline
I_{R4} & 0.001195\\ \hline
I_{R5} & -0.001238\\ \hline
I_{R6} & 0.000978\\ \hline
I_{R7} & 0.000978\\ \hline
V_1 & 5.125627\\ \hline
V_2 & 4.903891\\ \hline
V_3 & 4.446215\\ \hline
V_4 & 8.768409\\ \hline
V_5 & -2.982745\\ \hline
V_6 & -1.975719\\ \hline
V_7 & 4.934963\\ \hline

  \end{tabular}
  \caption{Operating point. A variable preceded by @ is of type {\em current}
    and expressed in Ampere; other variables are of type {\it voltage} and expressed in
    Volt.}
  \label{tab:op}
\end{table}

\lipsum[1-1]


\subsection{Transient Analysis}

Figure~\ref{fig:trans} shows the simulated transient analysis results for the
circuit under analysis. Compared to the theoretical analysis results, one
notices the following differences: describe and explain the differences.


\lipsum[1-1]



\subsection{Frequency Analysis}

\subsubsection{Magnitude Response}

Figure~\ref{fig:acm} shows the magnitude of the frequency response for the
circuit under analysis. Compared to the theoretical analysis results, one
notices the following differences: describe and explain the differences.

\lipsum[1-1]

\subsubsection{Phase Response}

Figure~\ref{fig:acp} shows the magnitude of the frequency response for the
circuit under analysis. Compared to the theoretical analysis results, one
notices the following differences: describe and explain the differences.


\lipsum[1-1]

\subsubsection{Input Impedance}

Figure~\ref{fig:zim} shows the magnitude of the frequency response for the
circuit under analysis. Compared to the theoretical analysis results, one
notices the following differences: describe and explain the differences.


\lipsum[1-1]



